
% Title
\newcommand{\pubtitle}{Anoma Resource Machine Specification}

\newcommand{\pubauthA}{Yulia Khalniyazova}
\newcommand{\pubaffilA}{a}
% \newcommand{\orcidA}{0000-0001-5477-1503}
\newcommand{\authemailA}{\{yulia, cwgoes\}@heliax.dev}
% \newcommand{\eqcontribA}{}

\newcommand{\pubauthB}{Christopher Goes}
\newcommand{\pubaffilB}{a}
% \newcommand{\orcidB}{0000-0001-0000-0000}
\newcommand{\authemailB}{cwgoes@heliax.dev}
% \newcommand{\eqcontribB}{}

% \newcommand{\pubauthC}{Last Author}
% \newcommand{\pubaffilC}{a}
% \newcommand{\orcidC}{0000-0001-5477-1503}
% \newcommand{\authemailC}{mail@someinstitute.com}

% Institutions/Affiliations
\newcommand{\pubaddrA}{Heliax AG}

% Corresponding author mail
\newcommand{\pubemail}{\authemailA}

\newcommand{\pubabstract}{
The article explores the Anoma Resource Machine (ARM) within the Anoma protocol, providing a comprehensive understanding of its role in facilitating state updates based on user preferences. Drawing parallels with the Ethereum Virtual Machine, the ARM introduces a flexible transaction model, diverging from traditional account and UTXO models. Key properties such as atomic state transitions, information flow control, account abstraction, and an intent-centric architecture contribute to the ARM's robustness and versatility. Inspired by the Zcash protocol, the ARM leverages commitment accumulators to ensure transaction privacy. The article outlines essential building blocks, computable components, and requirements for constructing the ARM, highlighting its unique approach to resource-based state management.
}

% Description of the SI file, placed as a footnote
% \newcommand{\pubSI}{Electronic Supplementary Information (ESI) available:
% one PDF file with all referenced supporting information.}

% Any keywords to be displayed under the abstract
\keywords{
Anoma Resource Machine \sep
resource model \sep
virtual machine \sep 
transaction privacy \sep
 
}

% Supplementary space between title/abstract and text, if needed
% \newcommand{\pubVadj}{0pt}

% ! DO NOT REMOVE OR MODIFY !
% Do not modify this file!

\title{\pubtitle}

\newcommand{\dg}{\textsuperscript{\textbf{\dag\textnormal{,}}}}

\ifdef{\pubauthA}{\author[\pubaffilA]{\pubauthA\ifdef{\orcidA}{~\protect\orcid{\orcidA}}{}\ifdef{\eqcontribA}{\dg}{}}}{}
\ifdef{\pubauthB}{\author[\pubaffilB]{\pubauthB\ifdef{\orcidB}{~\protect\orcid{\orcidB}}{}\ifdef{\eqcontribB}{\dg}{}}}{}
\ifdef{\pubauthC}{\author[\pubaffilC]{\pubauthC\ifdef{\orcidC}{~\protect\orcid{\orcidC}}{}\ifdef{\eqcontribC}{\dg}{}}}{}
\ifdef{\pubauthD}{\author[\pubaffilD]{\pubauthD\ifdef{\orcidD}{~\protect\orcid{\orcidD}}{}\ifdef{\eqcontribD}{\dg}{}}}{}
\ifdef{\pubauthE}{\author[\pubaffilE]{\pubauthE\ifdef{\orcidE}{~\protect\orcid{\orcidE}}{}\ifdef{\eqcontribE}{\dg}{}}}{}
\ifdef{\pubauthF}{\author[\pubaffilF]{\pubauthF\ifdef{\orcidF}{~\protect\orcid{\orcidF}}{}\ifdef{\eqcontribF}{\dg}{}}}{}
\ifdef{\pubauthG}{\author[\pubaffilG]{\pubauthG\ifdef{\orcidG}{~\protect\orcid{\orcidG}}{}\ifdef{\eqcontribG}{\dg}{}}}{}
\ifdef{\pubauthH}{\author[\pubaffilH]{\pubauthH\ifdef{\orcidH}{~\protect\orcid{\orcidH}}{}\ifdef{\eqcontribH}{\dg}{}}}{}
\ifdef{\pubauthI}{\author[\pubaffilI]{\pubauthI\ifdef{\orcidI}{~\protect\orcid{\orcidI}}{}\ifdef{\eqcontribI}{\dg}{}}}{}
\ifdef{\pubauthJ}{\author[\pubaffilJ]{\pubauthJ\ifdef{\orcidJ}{~\protect\orcid{\orcidJ}}{}\ifdef{\eqcontribJ}{\dg}{}}}{}
\ifdef{\pubauthK}{\author[\pubaffilK]{\pubauthK\ifdef{\orcidK}{~\protect\orcid{\orcidK}}{}\ifdef{\eqcontribK}{\dg}{}}}{}

\ifdef{\eqcontribA}{\equalcontrib{}}{}
\ifdef{\eqcontribB}{\equalcontrib{}}{}
\ifdef{\eqcontribC}{\equalcontrib{}}{}
\ifdef{\eqcontribD}{\equalcontrib{}}{}
\ifdef{\eqcontribE}{\equalcontrib{}}{}
\ifdef{\eqcontribF}{\equalcontrib{}}{}
\ifdef{\eqcontribG}{\equalcontrib{}}{}
\ifdef{\eqcontribH}{\equalcontrib{}}{}
\ifdef{\eqcontribI}{\equalcontrib{}}{}
\ifdef{\eqcontribJ}{\equalcontrib{}}{}
\ifdef{\eqcontribK}{\equalcontrib{}}{}

\ifdef{\pubaddrA}{\affil[a]{\pubaddrA}}{}
\ifdef{\pubaddrB}{\affil[b]{\pubaddrB}}{}
\ifdef{\pubaddrC}{\affil[c]{\pubaddrC}}{}
\ifdef{\pubaddrD}{\affil[d]{\pubaddrD}}{}
\ifdef{\pubaddrE}{\affil[e]{\pubaddrE}}{}
\ifdef{\pubaddrF}{\affil[f]{\pubaddrF}}{}
\ifdef{\pubaddrG}{\affil[g]{\pubaddrG}}{}
\ifdef{\pubaddrH}{\affil[h]{\pubaddrH}}{}

\contact{\pubemail}

%% Abstract
%% --------

\begin{abstract}
    \pubabstract{}
\end{abstract}

%% Keywords
%% --------

\ifdef{\pubkeywords}{\keywords{\pubkeywords}}{}

%% Adjusting vertical space
%% ------------------------

\ifdef{\pubVadj}{\verticaladjustment{\pubVadj}}{}

% The preprint DOI to be used as an link in the paper
\pubdoi{10.5281/zenodo.10498991}
\history{(Received Nov 10, 2023; Revised Jan 25, 2024 ; Version: Jan 25, 2024)}